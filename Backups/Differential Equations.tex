\documentclass[11pt]{article}
\author{Tarang Srivastava}
\usepackage{amsmath}
\usepackage{amsthm}
\usepackage[margin=1in]{geometry}
\theoremstyle{definition}
\newtheorem{theorem}{Example}
\theoremstyle{definition}
\newtheorem{definition}{Definition}[section]

\begin{document}
	\begin{titlepage}
	    	\begin{center}
		        \vspace*{1cm}
		        \Huge
		        \textbf{Differential Equations}
		      		
		        \vspace{0.5cm}
		        \LARGE
		        A Concise Review of Elementary Differential Equations
		        
		        \vspace{1.5cm}
		        
		        \textbf{Tarang Srivastava}
		        
		        \vfill
		        
		        \vspace{0.8cm}
		        
		        %\includegraphics[width=0.4\textwidth]{university}
		        
		        \Large
		        Dr. Cakir\\
		        Differential Equations and Complex Analysis\\
		        South Brunswick High School\\
		        14 February 2018
	        
	    \end{center}
	\end{titlepage}
	\tableofcontents	
	\pagebreak
	
	\section{Introduction}
		\subsection{Direction Fields}	
			Equations containing derivatives are differential equations. Differential Equations can be modeled by directional field (slope field). Directional fields are given in the form 
			$$\dfrac{dy}{dt} = f(t,y)$$
			<Insert image here> 
			Directional fields can be used to find the equilibrium solution. 
		\subsection{Solutions of Some Differential Equations}
			Consider the following differential equation
			\begin{theorem} %Example 1
				\begin{align}
					\dfrac{dp}{dt} = \dfrac{p-900}{2} \\ 
					\dfrac{dp / dt}{p-900} = \dfrac{1}{2} \\
					\dfrac{d}{dt} \ln | p - 900 | = \dfrac{1}{2} \\
					\intertext{Integrate both sides}
					 \ln|p-900| = t/2 + C\\
					 |p-900| = e^C e^{t/2}\\
					 p-900 = \pm e^C e^{t/2}\\
					\intertext{Note how $e^C$ is simply a constant c and the $\pm$ can be replaced by the c}
					 p = 900 + ce^{t/2}
				\end{align}
					The solution to this differential equation is given as such where c is any non zero constant
			\end{theorem}
				The general solution can be found using a similar method 
			\begin{definition}{General Solution}
				\begin{align}
					\intertext{given a general differential equation}
					\dfrac{dy}{dt} = ay - b \\
					\intertext{with initial condition}
					y(0) = y_0\\
					\dfrac{dy}{dt} = (y - b/a)a\\
					\dfrac{dy/dt}{y-b} = a\\
					\intertext{Integrating both sides}
					ln|y-(b/a)| = at + C \\
					y = b/a + ce^{at}
					\intertext{Given the initial condition C can be replaced, giving}
					 y = (b/a) + [y_0 - (b/a)]e^{at}
				\end{align}
			\end{definition}

\end{document}
